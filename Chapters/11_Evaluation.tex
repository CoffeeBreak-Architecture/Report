\chapter{Evaluation and Reflections}

This chapter discusses the the project and how successful and satisfying it has been. Most of this chapter is dedicated to discussing the faults and failures of the development process, where things went wrong, what went wrong, what could have been done alternatively, or what could have been done to avoid it. The chapter is split into a section for each step of development corresponding to all chapters between this and Requirements, as well as a section on the impact of the Covid-19 quarentine.

\section{Requirements}

Developing the initial set of requirements were without too many issues. After the various proposed projects were presented for us, a number of teams chose the CoffeeBreak project, and none of us were really sure if all the teams were supposed to work together, or if we should make individual projects. This did somewhat slow down the initial requirement elicitation, though after debating the requirements with the other groups we did end up with what we felt, and still feel, is a fairly robust set of system requirements. While we are satisfied with the final result, the requirement elicitation did occupy our minds for longer than they needed to, at times where other initial project aspects might have needed attention.

As a result of this, we had little mental room to figure out the core academic purpose of the project early, preventing us from generating a solid footing and performing a good and proper literature review. This lack of a solid project core has affected the project throughout, mostly as the core we finally did find in the middle of the project, the measuring of performance of different implementation variations, was without much academic backing and without later potential consequences truly thought out. With this lesson learned the hard way, we hope to keep a good academic core in mind for future academic projects.

\section{Analysis}

Establishing a finalized requirement list was straight forward due to the already thoroughly established overall requirements. Creating use cases to further analyse the different system interactions gave us a deeper insight as to what was required. Frequent feedback reports from our supervisors during this phase of the project development, further gave us information in regards to how specific the requirements established was and if they needed adjusting or narrowed down. This feedback and close relationship between us and our supervisors gave us a solid requirement specification of which to work from in the design phase and beyond. 

The analysis itself went alright, the resulting finalized requirement specification, as well as the throughout exploration of the potential system through detailed use case specifications did help us find the initial design which to some extend survived towards the end of the project. We are, however, uncertain if the requirements are specific and clear enough that they could be given to another team to implement correctly according to our vision. Making greater use of behaviour-driven development principles, in particular object-oriented analysis, might result in a clearer, more concise requirement specification, with less implementation-specific information scattered throughout, as well as more consistent levels of detail where needed.

\section{Design}

The design suffered heavy simplification through development. The impracticality of some of the originally proposed design became particularly clear. For instance, the originally proposed design had individual Kubernetes pods per rooms or per users, but during implementation it became clear that this design would become highly over complicated, and a much simpler design would work much better with Kubernetes. Looking back, we should have further investigated the different technologies we wanted to incorporate, and had a better understanding of their capabilities before we decided how we wanted to use them. This could potentially have been avoided by trying to build a simple prototype of the originally proposed design. At least then we would have been better prepared for the issues with the design, and might have been better at choosing more compatible technologies. While we are happy with the final design in the context of the product, the academic core of the project relied heavily upon it, and as a result suffered tremendously.

\section{Implementation}

The analysis and design phases gave a good overall overview of the systems architecture initially, and the initial few days of implementation started out looking at the feasibility of the design, by working with the technologies hands on. However when working with the different technologies hands-on, it gave a more thorough understanding of their strengths and weaknesses, in addition to what was possible within the design already established. This caused a lot of iterative changes to the design in order to implement certain features and resulted in a much different design of the system than initially thought out. The overall functionality was still implemented, and many of the established requirements were fulfilled, however many also had to be removed to fit the new design that was evolved during the implementation.

The implementation phase was an intensive period of making a functioning system. Due to the above problems, and no time to go back to the drawing board and recreating the design, "on the fly" solutions were the answer to most if not all problems that occurred during the process. Overall we think that the implementation phase, given the circumstances went well. If more time were available, a second iteration of analysis and design would be fitting, in order to adjust with the experience in mind. 

\section{Verification}

There is little to complain about with the process of verifying the system requirements. Some requirements turned out to be difficult to validate properly and automatically due to the complexities of doing so, or the lack of testable related code. In general the code was fairly, but not perfectly testable. While test-driven development principles were discussed early in development, we decided against it due to the development overhead commonly entailed, though it would likely have improved the quality and quantity of verification significantly. Most testable units as well as interactions between components have been tested, though not as thoroughly as they could be, as there is for the most time only a single test for each unit, with an occasional negative test to further verify. Optimally, each unit should have one positive test, one negative test, and one exception test in our opinion.

CI using GitHub Actions worked without significant hassle. There was some difficulty getting true integration testing working, and as a result only a proof-of-concept was done for a single component. Aside from this, GitHub Actions has proven a strong contender amongst other CI services.

\section{Deployment}
The initial deployment detail was early in development and went through many different iterations and complications, before finally finding a solution that solved all of our problems. These early problems caused the time it took to get a fully working deployment to increase massively and cause certain goals to be deemed unreachable due to time constraints. One of these was a fully deployed Kubernetes cluster containing the MRS solution, Vili. Due to the aforementioned time constraints, a proper Kubernetes deployment of the Vili solution never came to light. 

The Vili solution, if implemented, would consist of a Media Relay Server that would represent each room. All media would have to go through the Media Relay Server instead of going straight to the users. The Kubernetes Cluster would essentially function the same, however room creation would rely heavily on a Kubernetes API implementation in the room manager. When a room is created a new MRS Kubernetes deployment would be created specifically for the room. This deployment would then be connected to from the client.

Unfortunately this never came to life, and the furthest extent of a proper deployment is a docker-compose of all the services working, with several aspects of the aforementioned proposed solution not implemented yet. However this solution to a deployment works flawlessly on a local machine, so showcasing the would be system is a possibility

\section{Results}

The results of the project is lackluster to say the least due to reasons mentioned in the design section of this chapter. With the hardship of finding a proper angle, it was tough to establish a proper aspect to assess the project. In the end, the result ended with being the established product. This might be acceptable in the context of primary goal established for this project, however when reflecting, it pains us that we couldn't establish a more academic angle and goal to research and answer. 

In search of a more measurable goal and potential result, we established a secondary goal that consists of performance analysis of our two implemented solutions. However due to this being a secondary goal, the main focus was on creating a finished product that fulfilled our main goal. This caused the results gained from this to be thin and not measured in certain aspects either due to not having implemented certain features that makes measurement possible, which again is lacking due to time constraints. However the results gained, albeit small, align with the different WebRTC configurations in the Design phase. With more thorough testing on, it could have given us more solid data to display.

\section{Impact of the Covid-19 Quarentine}

As may be expected from any project developed during the Covid-19 pandemic between 2020 and 2021, the project has suffered somewhat due to the mental health impact of quarantine. The team suffered from a lack of motivation at times, and aspects of the project suffered, particularly where the team lacks significant skill such as in analysis. We believe that the project foundation would have been significantly stronger if, what is colloquially known as "Quarantine Depression" had not had such a mental toll, but such is life sometimes. The worst of this was several times late in development where the team suffered severe burnouts, possibly due to the lack of daily structure, as well as an unhealthy work-life balance. This ultimately resulted in some lack of intended features and polish, but we feel that we got decently enough through it after all.

\section{Conclusion of Evaluation}

To say any project goes perfectly to plan would probably be a lie, and this project is no different. The greatest challenges faced with this project were difficulties getting truly going with a solid core. This affected large parts of the system, as a core had to be found halfway through, resulting in one that was somewhat poorly thought out. To further worsen this, the original design intended to work with this core turned out impractical, and was simplified to be much more practical, at the cost of detailed academic results. While the Covid-19 pandemic did take a heavy toll at times, and lessened the quality of the project, more severely in some aspects than others, we feel that we made it through well enough regardless.

Aside from that, each part of development each faced various issues, though we do not have particularity severe regrets about most other than analysis, which could have been significantly better using more robust principles.