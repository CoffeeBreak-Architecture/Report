\section*{Abstract}
\addcontentsline{toc}{section}{Abstract}

\blindtext

Online communications become more and more prevalent these days, even more so during the Covid-19 pandemic. There are multiple common ways of facilitating online communications, some with direct connections between peers, some where the connections go through a relay between the peers.

This project aims to implement an online communication service that can help users have immersive online communication with each other in a virtual space. The goal is to implement a robust and scalable solution that can deliver on the above aspects and provide at good platform for communication. This will be implemented based on two different designs, in order to measure the performance difference between the two common solutions. In order to accomplish this, each solution is deployed on the same hardware, and hardware utilization is measured while increasing load. The result ended with a fully deployed communication system that enables spacially aware voice communication. Additionally, the resulting product and the tests performance, points towards that, at least for the client, that a relay between peers is a significantly better solution. This is due to linear increase in hardware utilization, in contrast to an exponential increase of connections directly between peers. However, the relay-based solution has greater complexity, and can be significantly more difficult to implement.

While it is easily tempting for a new, naive startup to think they should develop a similar system to allow as many people as absolutely possible through the best possible solutions, the complexities of the most difficult solutions can easily be prohibitive, and result in the final system actually being of worse quality than if they had chosen a simpler, yet worse performing solution. However disregarding these results, a fully functioning system was created based on the peer to peer design that upholds all established requirements set for the project. 