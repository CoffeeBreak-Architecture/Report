\chapter{Learning Objectives}

The primary learning objectives for each team member has been varied, but both shared the goal of gaining a better understanding of the complete process of product development, in particular using service-oriented architectures and containerization technologies. Additionally, the team has worked to improve their report writing abilities as both has struggled somewhat with this in previous projects, and a project where each person has much greater overview of the project as a whole, as well as a greater feeling of product ownership, means that the project is a great opportunity for improving writing and communication skills.

Additionally, the team has been interested in the benefits of a highly agile development process compared to previous projects, that, while drawing the majority of inspiration from the Agile Iterative Process, does away with most unnecessary fluff and only use the parts that might actually be of use to a two-person team. Furthermore, the team wishes to try out the concept of Domain Driven Design.

Individually, Frederiks personal goals has been to deep dive into the Kubernetes and figure out how to set up a fully functioning deployment, from the individual pod and deployment configurations, to overall load balancing, deployment and maintenance on PaaS providers such as Google or Amazon. Additionally, he was intrigued by the possibility of adjusting and modifying an existing Kubernetes programmatically. Lastly, he wanted to further improve his knowledge of WebRTC, by investigating its different implementation designs and their benefits and flaws. 

On the other hand, Marcus' personal goals has been to greatly expand upon his previously rather lackluster web development skill set, in particular by expanding HTML and CSS knowledge, and by working with Node.js and Socket.io to implement the back-end components for the system. Furthermore, he has yearned to improve his understanding of the design of service-oriented architectures, in particular microservice-based architectures. Finally, he wished to better his understanding of RESTful API design, and API design in general. All in an attempt to grow to a more well-rounded full-stack developer.